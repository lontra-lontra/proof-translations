
\def\structureDefinition{
\begin{definition}
    \textbf{A Structure of a Language} consists of a domain and:
    \begin{itemize}
        \item An assignment from the constant symbols of the language to the domain.
        \item An assignment from the predicate symbols of the language to predicates in the domain.
    \end{itemize}
\end{definition}
}

\def\kripkeFrameDefinition{
\begin{definition}
    \textbf{A Kripke Frame} of a Language \(\mathcal{L}\), \(\mathcal{C} = (R, \{C(p)\}_{p \in R})\) consists of a partially ordered set \(R\), and an \(\mathcal{L}\)-structure \(C(p)\) for all \(p\)'s in \(R\). Furthermore, in a Kripke Frame, if \(p \leq q\), then \(C(q)\) extends \(C(p)\):
    \begin{itemize}
        \item All sentences that are true in \(C(p)\) are true in \(C(q)\).
        \item The domain of \(C(p)\) is included in the domain of \(C(q)\).
        \item The assignments in \(C(p)\) are the same as in \(C(q)\) for the domain in common.
    \end{itemize}
\end{definition}
}

\def\forcingDefinition{
\begin{definition}
    \textbf{Forcing}. When a sentence \(\phi\) of a language \(\mathcal{L}\) is \textbf{forced} by a structure \(C(p)\) of a \(\mathcal{L}\)-frame \(\mathcal{C}\), we denote:
    \(p \Vdash_{\mathcal{C}} \phi\) \\
    Forcing is defined by induction: \cite{book1}
    \begin{itemize}
        \item \(p \Vdash_{\mathcal{C}} \phi \Leftrightarrow \phi \text{ is true in } C(p)\) (if \(\phi\) is an atomic sentence).
        \item \(p \Vdash_{\mathcal{C}} (\phi \to \psi) \Leftrightarrow\) for all \(q \geq p\), if \(q \Vdash_{\mathcal{C}} \phi\), then \(q \Vdash_{\mathcal{C}} \psi\).
        \item \(p \Vdash_{\mathcal{C}} \neg \phi \Leftrightarrow\) for all \(q \geq p\), \(q\) does not force \(\phi\).
        \item \(p \Vdash_{\mathcal{C}} (\forall x) \phi(x) \Leftrightarrow\) for all \(q \geq p\) and \(d\) in \(\mathcal{L}_{C(q)}\), \(q \Vdash_{\mathcal{C}} \phi(d)\).
        \item \(p \Vdash_{\mathcal{C}} (\exists x) \phi(x) \Leftrightarrow\) there exists a \(d\) in \(\mathcal{L}_{C(q)}\), such that \(p \Vdash_{\mathcal{C}} \phi(d)\).
        \item \(p \Vdash_{\mathcal{C}} (\phi \land \psi) \Leftrightarrow p \Vdash_{\mathcal{C}} \phi \text{ and } p \Vdash_{\mathcal{C}} \psi\).
        \item \(p \Vdash_{\mathcal{C}} (\phi \lor \psi) \Leftrightarrow p \Vdash_{\mathcal{C}} \phi \text{ or } p \Vdash_{\mathcal{C}} \psi\).
    \end{itemize}
\end{definition}
}

\def\intuitionisticValidityDefinition{
\begin{definition}
    \textbf{Intuitionistic Validity}. A sentence of a language \(\mathcal{L}\) is Intuitionistically valid if it is forced in all structures of all Kripke frames of \(\mathcal{L}\).
\end{definition}
}

\def\classicalValidityDefinition{
\begin{definition}
    \textbf{Classical Validity}. A sentence of a language \(\mathcal{L}\) is classically valid if it is forced by all single-structure Kripke frames of \(\mathcal{L}\).
\end{definition}
}

\def\signedSentenceIntuitionisticDefinition{
\begin{definition}
    \textbf{A Signed Sentence} is a forcing assertion inside of a tableaux proof. It looks like $T_{q} \phi $ or $F_{p} \phi $ \\
    \textbf{A Signed Sentence List} is a list forcing assertions inside of a tableau proof. We say that a list of forcing assertions having sentences $\{ T_{p_1}\gamma_{1}, T_{p_2}\gamma_{2}, ...\}$
  and $ \{F_{q_1}\delta_{1}, F_{q_2}\delta_{2}...\} $
     is "intuitionistically valid" (Question 2: maybe use a term != valid?)if there exists a frame $\mathcal{C}$ for which $ \mathcal{C}(p_1) \Vdash \gamma_{1}$ and $ \mathcal{C}(p_{2}) \Vdash \gamma_{2}$ and ... $ \mathcal{C}(q_{1}) \nvDash \delta_{1}$  and $ \mathcal{C}(q_{2}) \nvDash \delta_{2}$
\end{definition}
}
\def\fIntuitionisticDefinition{
\begin{definition}
    The function $f$ takes a signed sentence $\sigma$ and a signed sentence list $L$ and returns one or two signed sentence lists.  \\$f(\sigma,L)$ is defined as follows:
    \\(here we denote $l||l'$ = $l_1,l_2...,l_{|l|}, l'_1,l'_2..,l'_{|l'|} $)
    \\if $\sigma \in L$:
        \begin{itemize}
            \item $f(T_{p} \neg \alpha,L) = [L - \sigma|| \sigma || F_{p'}\alpha] $ \\for a minimal $p' \geq p$ present in $h||L$.
            \item $f(F_{p} \neg \alpha,L) = [L - \sigma || \sigma || T_p' \alpha]$  \\for a new $p' \geq p$ 
            \item $f(T_{p} (\alpha \land \beta), L) = [L - \sigma|| \sigma || T_{p} \alpha || T_{p} \beta]$.
            \item $f(F_{p} (\alpha \land \beta), L) = [L - \sigma|| \sigma || F_{p} \alpha],[ L - \sigma|| h || F_{p} \beta]$.
            \item $f(T_{p} (\alpha \lor \beta), L) = [L - \sigma|| \sigma || T_{p} \alpha],[ L - \sigma|| h || T_{p} \beta]$.
            \item $f(F_{p} (\alpha \lor \beta), L) = [L - \sigma|| \sigma || F_{p} \alpha || F_{p} \beta]$.
            \item $f(T_{p} (\alpha \to \beta), L) = [L - \sigma|| \sigma || F_{p'} \alpha],[ L - \sigma || h || T_{p'} \beta]$ \\for a new $p' \geq p$ 
            \item $f(F_{p} (\alpha \to \beta), L) = [L - \sigma|| \sigma || T_{p'} \alpha || F_{p'} \beta]$ \\for a minimal $p' \geq p$ present in $h||L$.
            \item $f(T_{p} (\forall x) \phi(x), L) = [L - \sigma|| \sigma || T_{p} \phi(c_i) ]$\\ for the first constant $c_i$ for which $T_{p} \phi(c_i)$ is not in $L$.
            \item $f(F_{p'} (\forall x) \phi(x), L) = [L - \sigma|| \sigma || F_{p} \phi(c_i) ]$\\ for the first constant $c_i$ not present in $h||L$ and a new $p' \geq p$
            \item $f(T_{p'} (\exists x) \phi(x), L) = [L - \sigma|| \sigma || T_{p} \phi(c_i) ]$ \\for the first constant $c_i$ not present in $h||L$ and a new $p' \geq p$.
            \item $f(F_{p} (\exists x) \phi(x), L) = [L - \sigma|| \sigma || F_{p} \phi(c_i) ]$\\ for the first constant $c_i$ such that $F_{p} \phi(c_i)$ is not in $L$ and .
        \end{itemize} [TODO revise: a stronger criterium for contradiction makes some of these useless. in any case, this "simetric" form is  more easy to implement ]\\
    if $\sigma \notin L$:
    \begin{itemize}
         \item $f(\sigma,L) = L $
    \end{itemize}
\end{definition}
}


\def\wellBehavedTheorem{
    \begin{theorem}
        Given a signed sentence list x, if x is intuitionistically valid then f(x) is intuitionistically valid 
    \end{theorem}
    \begin{proof}
        \cite{book1} proves this by using the definition of forcing. The choices of $p'$ and $c_i$ garantes the completnees of the tableaux.

        Case: if $L \ni F_{p} (\forall x) \phi(x) $ is intuitionistically valid, then there exists frame $\mathcal{C}$ that "respects" it. We take 
        a frame $\mathcal{C}'$ that is exactly like $\mathcal{C}$ with adition of a structure $p' \geq p$ such that $\mathcal{C}'(p') \nVdash \phi(c_i)$ for a new $c_i$. We now that 
        $\mathcal{C}'$ exists by the defintion of forcing for $\forall$. Also $\mathcal{C}'$ is a frame that "respects"
        $f(\forall x \phi(x), L) = [L - \sigma|| \sigma || T_{p} \phi(c_i) ]$
    \end{proof}
}

\def\TableauxDevelopmentDefinition{
    \begin{definition}
        \textbf{The tableaux Development of a Sentence } is defined inductivelly: 
        \begin{itemize}
            \item A tree with the single node $F_{\emptyset}\phi$ is a tableaux development of $\phi$. 
            \item If $\tau$ is a tableaux development of $\phi$, then $\hookleftarrow(\sigma,\tau)$ is a tableaux development of $\phi$. Where:
        \end{itemize}
    
        $\hookleftarrow(\sigma,\tau) = $ $\tau$ with $f(\sigma, l) $added to all leaves l that contain $\sigma$
    \end{definition}
}

\def\TableauxDevelopmentListDefinition{
    \begin{definition}
        \textbf{The tableaux Development of a List of Sentences } is defined inductivelly: 
        \begin{itemize}
            \item A tree with the single node $L$ is a tableaux development of $L$. 
            \item If $\tau$ is a tableaux development of $L$, then $\hookleftarrow(\sigma,\tau)$ is a tableaux development of $\phi$. Where:
        \end{itemize}
    
        $\hookleftarrow(\sigma,\tau) = $ $\tau$ with $f(\sigma, l) $added to all leaves l that contain $\sigma$
    \end{definition}
}



\def\soundnessTheorem {
\begin{theorem}
    If $F_{\emptyset}\phi$ is intuitionistically valid then one of the leaves of $ \hookleftarrow (\sigma_n,\hookleftarrow(\sigma_{n-1}(.....(\hookleftarrow(\sigma_1 ,F_{\emptyset}\phi )...)))) $ is intuitionistically valid.
\end{theorem}

\begin{proof}

    The proof goes by induction:


             The base case is true by definition: either the root $F_{\emptyset}\phi$ is a intuitionistically valid leaf or the premisse is false.  Next we assume $\tau' = \hookleftarrow (\sigma, \tau)$.
             There are two cases to consider:
            \begin{itemize}
                \item If there does not exist a frame that does not force $\phi$, $F\phi$ is not intuitionistically valid and the theorem holds for $\tau'$.
                \item  If there exists a frame that does not force $\phi$:

                         By the induction hypothesis, there exists a intuitionistically valid leaf $\sigma$ in $\tau$.
                         \begin{itemize}
                    \item If $\sigma$ is a leaf in $\tau$, then $F_{\emptyset}\phi$ the theorem holds for $\tau'$.
                    \item If $\sigma$ is not a leaf in $\tau$, then:
                         One or two nodes were added to $\sigma$ in $\tau'$ by the definition of $\hookleftarrow$.
                         By the theorem 1, one of the added nodes is also intuitionistically valid.
                         Consequently, the theorem holds for $\tau'$.

            \end{itemize}
            \end{itemize}
\end{proof}

}

\def\completeiteratordefinition{
\begin{definition}
    $\hookleftarrow_c(\tau) = \tau$ with $f(h,l)$ added to all leaves l that contain $\sigma$. h is the first signed sentence of the shalowest (and after that leftmost) non-contradictory leaf. 
\end{definition}
}


\def\signedSentenceClassicalDefinition{

\begin{definition}
    \textbf{A Signed Sentence} is a forcing assertion inside of a tableaux proof. It looks like $T \phi $ or $F \phi $ \\
    \textbf{A Signed Sentence List} is a list forcing assertions inside of a tableau proof. We say that a list of forcing assertions having sentences $\{ T\gamma_{1}, T\gamma_{2}, ...\}$
  and $ \{F\delta_{1}, F\delta_{2}...\} $
     is "classically valid" (Question 2: maybe use a term != valid?)if there exists a single-structured frame  $\mathcal{C}$ such that $ \mathcal{C}(\emptyset) \Vdash \gamma_{1}$ and $ \mathcal{C}(\emptyset) \Vdash \gamma_{2}$ and ... $ \mathcal{C}(\emptyset) \nvDash \delta_{1}$  and $ \mathcal{C}(\emptyset) \nvDash \delta_{2}$
\end{definition}
 
}

\def\fClassicalDefinition{
    \begin{definition}
        The function $f$ takes a signed sentence $\sigma$ and a signed sentence list $L$ and returns one or two signed sentence lists.  \\$f(\sigma,L)$ is defined as follows:
        \\(here we denote $l||l'$ = $l_1,l_2...,l_{|l|}, l'_1,l'_2..,l'_{|l'|} $)
        \\if $\sigma \in L$:
            \begin{itemize}
                \item $f(T \neg \alpha,L) = [L - \sigma|| \sigma || F \alpha]$.
                \item $f(F \neg \alpha,L) = [L - \sigma || \sigma || T \alpha]$.
                \item $f(T (\alpha \land \beta), L) = [L - \sigma|| \sigma || T \alpha || T \beta]$.
                \item $f(F (\alpha \land \beta), L) = [L - \sigma|| \sigma || F \alpha],[ L || \sigma || F \beta]$.
                \item $f(T (\alpha \lor \beta), L) = [L - \sigma|| \sigma || T \alpha],[ L || \sigma || T \beta]$.
                \item $f(F (\alpha \lor \beta), L) = [L - \sigma|| \sigma || F \alpha || F \beta]$.
                \item $f(T (\alpha \to \beta), L) = [L - \sigma|| \sigma || F \alpha],[ L || \sigma || T \beta]$.
                \item $f(F (\alpha \to \beta), L) = [L - \sigma|| \sigma || T \alpha || F \beta]$.
                \item $f(T (\forall x) \phi(x), L) = [L - \sigma|| \sigma || T \phi(c_i)]$ \\ for the first constant $c_i$ for which $T \phi(c_i)$ is not in $L$.
                \item $f(F (\forall x) \phi(x), L) = [L - \sigma|| \sigma || F \phi(c_i)]$ \\ for the first constant $c_i$ not present in $L$.
                \item $f(T (\exists x) \phi(x), L) = [L - \sigma|| \sigma || T \phi(c_i)]$ \\ for the first constant $c_i$ not present in $L$.
                \item $f(F (\exists x) \phi(x), L) = [L - \sigma|| \sigma || F \phi(c_i)]$ \\ for the first constant $c_i$ for which $F \phi(c_i)$ is not in $L$.
            \end{itemize} [TODO revise] \\
        if $\sigma \notin L$:
        \begin{itemize}
             \item $f(\sigma,L) = L $
        \end{itemize}
    \end{definition}
}


\def\sequentDefinition{    
\begin {definition}
    \textbf{A Sequent}  is an expression of the form 
    \[
    \Gamma \vdash \Delta
    \]
    where $\Gamma = \{\Gamma_1, \Gamma_2, \Gamma_3,...\}$ and $\Delta = \{\Delta_1,\Delta_2,\Delta_3 ...\}$
     are  finite sets of formulas. $\Gamma $ is called the antecedent, and 
    $\Delta$ is called the succedent. 
\end {definition}
}

\def\sequentProofDefinition{    
\begin {definition}
    \textbf{Sequent} A sequent is an expression of the form 
    \[
    \Gamma \vdash \Delta
    \]
    where $\Gamma = \{\Gamma_1, \Gamma_2, \Gamma_3,...\}$ and $\Delta = \{\Delta_1,\Delta_2,\Delta_3 ...\}$
     are  finite sets of formulas. $\Gamma $ is called the antecedent, and 
    $\Delta$ is called the succedent. 
\end {definition}
}


\def\sequentValidityDefinition{
    \begin {definition} \textbf{Intuitionistical Validity of a Sequent}
A sequent is intuitionistically valid if for all kripke frames, if $\Gamma$ are forced, then at least one of $\Delta$ is forced.
\end {definition}

}




\def\developmentSequentIntuitionisticDefinition{

\begin {definition}
    A single sequent is a intuitionistic sequent proof tree development\\
    Any of the rules of the table 1 aplied to a intuitionistic sequent proof tree development is a sequent proof tree development\\
    A intuitionistic squent proof tree is a intuitionistic sequent proof tree development with axioms on all leafs.
\end {definition} 

}

\def\developmentSequentClassicalDefinition{

\begin {definition}
    A single sequent is a classical sequent proof tree development\\
    Any of the rules of the table 1 applied to a classical sequent proof tree development is a sequent proof tree development\\
    A classical sequent proof tree is a classical sequent proof tree development with axioms on all leaves.
\end {definition} 

}




\def\nodeTranslationFunction{
    \begin{definition} \textbf{Node Translating Function $\mathcal{T}$}


        Given a signed sentence list L with sentences
         $\{ T_{p_1}\gamma_{1}, T_{p_2}\gamma_{2}, ...\}$
        and 
            $ \{F_{q_1}\delta_{1}, F_{q_2}\delta_{2}...\}$ and a world $w \in \{p_1,p_2,...\} \cup \{q_1,q_2,...\}$ 
        then: \\  $ \mathcal{T} (L,w$) = $\Gamma_{1}, \Gamma_{2}, ... \vdash \Delta_1, \Delta_2, ...$, were: \\
        $\{ T_{p'_1}\Gamma_{1}, T_{p'_2}\Gamma_{2} , ...\}$    are the elements of $\{ T_{p_1}\gamma_{1}, T_{p_2}\gamma_{2}, ...\}$ such that  $p' \geq w$ 
        and 
        $\{ F_{q'_1}\Delta_{1}, F_{q'_2}\Delta_{2} , ...\}$ are the elements of $\{ F_{q_1}\delta_{1}, F_{q_2}\delta_{2}, ...\}$ such that  $q' \leq w$ 
        

        For the classical case: $ \mathcal{T} (L) = \mathcal{T} (L,\emptyset) $
        \end{definition}
}

\def\uselessTheorem{
    \begin{theorem}
        Given signed sentence list $L$ and a $w$,  
        if L is intuitionistically valid then the sequent $\mathcal{T}(L, w)$ is not intuitionistically valid.
    \end{theorem}
    
        \begin{proof} 
        If L has the sentences
        $\{ T_{p_1}\gamma_{1}, T_{p_2}\gamma_{2}, ...\}$
       and 
           $ \{F_{q_1}\delta_{1}, F_{q_2}\delta_{2}...\}$ ,it exists a frame $ \mathcal{C}$ such that: 
    
    
        $ \mathcal{C}(p_1) \Vdash \gamma_{1}$ and $ \mathcal{C}(p_{2}) \Vdash \gamma_{2}$ and ... $ \mathcal{C}(q_{1}) \nvDash \delta_{1}$  and $ \mathcal{C}(q_{2}) \nvDash \delta_{2}$ 
       \\  which implicitly means, by the definition of extension: 
        
       ($ \mathcal{C}(p) \Vdash \gamma_{1}$ for all $p \geq p_1$) and ($ \mathcal{C}(p) \Vdash \gamma_{2}$ for all $p \geq p_2$) and ... ($ \mathcal{C}(p) \nvDash \delta_{1}$ for all $p \leq q_{1}$) and ($ \mathcal{C}(p) \nvDash \delta_{2}$ for all $p \leq q_{2}$)...
         
       
       Take the structure $\mathcal{C}(w)$ inside of $\mathcal{C}$. We can infer:
       
       $ \mathcal{C}(w) \Vdash \Gamma_{1}$ and $ \mathcal{C}(w) \Vdash \Gamma_{2}$ and ... $ \mathcal{C}(w) \nvDash \Delta_1$  and $ \mathcal{C}(w) \nvDash \Delta_2 ...$.
       
       $\mathcal{C}(w)$ is thus a counterexample proving the non-validity of 
        $\Gamma_{1}, \Gamma_{2}, ... \vdash \Delta_1, \Delta_2, ...$
        \end{proof}
}

\def\otheruselesstheorem{
    \begin{theorem} 
        L is valid if and only if $\mathcal{T}(L,\emptyset)$ is not valid 
    \end{theorem}
    
    \begin{proof}
        It goes directly from the definiton:
        \begin{itemize}
    \item If the signed sentence list is valid, then the sequent is invalid by corolary of [TODO]
    \item If the signed sentence list is not valid, it does not exist a counterexample for $\mathcal{T}$(L). (Note that 
    we can not say the same for the intuitionistic case)
       \end{itemize}
    \end{proof}
}

\def\localTranslationValidityTheorem{
    \begin{theorem}
        Given a classical signed sentence list l with $\sigma \in l$ , then 
       $\RightLabel{}\AxiomC{$\mathcal{T}( f(\sigma,l))$}\UnaryInfC{$\mathcal{T}(l)$}\DisplayProof$ 
       is a classically valid rule.
   
   \end{theorem}
   
   \begin{proof}
       Here will show the implicit contradiction rule being used:\\ 
       if $L-\sigma$ has sentences $\{ T\Gamma_{1}, T\Gamma_{2}, ...\}$ and  $ \{F\Delta_1, F\Delta_2 ...\} $, then: 
       \small{
       \begin {itemize}
   
       \item
       $\AxiomC{$\mathcal{T}( f(T\neg \alpha,l))$}\UnaryInfC{$\mathcal{T}(L)$}\DisplayProof$ 
       $=\AxiomC{$\mathcal{T}( (L-T\neg \alpha)||T \neg \alpha||F \alpha)$}\UnaryInfC{$\mathcal{T}(L)$}\DisplayProof$
       
       
       $=\AxiomC{$\Gamma ,\neg \alpha \vdash \alpha, \Delta$}\UnaryInfC{$\Gamma, \neg \alpha \vdash \Delta$}\DisplayProof$ 
       $=\RightLabel{\scriptsize{$\neg$R}}\AxiomC{$\Gamma ,\neg \alpha \vdash \alpha, \Delta$}\UnaryInfC{$\Gamma, \neg \alpha , \neg \alpha \vdash \Delta$}\UnaryInfC{$\Gamma, \neg \alpha \vdash \Delta$}\DisplayProof$ 
       \item 
       $\AxiomC{$\mathcal{T}( f(F\neg \alpha,l))$}\UnaryInfC{$\mathcal{T}(L)$}\DisplayProof$ 
       $=\AxiomC{$\mathcal{T}( (L-F\neg \alpha)||F \neg \alpha||T \alpha)$}\UnaryInfC{$\mathcal{T}(L)$}\DisplayProof$
       
       
       $=\AxiomC{$\Gamma ,\neg \alpha \vdash \neg \alpha, \Delta$}\UnaryInfC{$\Gamma \vdash , \neg \alpha \Delta$}\DisplayProof$ 
       $=\RightLabel{\scriptsize{$\neg$L}}\AxiomC{$\Gamma , \alpha \vdash \neg \alpha, \Delta$}\UnaryInfC{$\Gamma \vdash \neg \alpha , \neg \alpha \Delta$}\UnaryInfC{$\Gamma,  \vdash  \neg \alpha \Delta$}\DisplayProof$ 
       \item 
       $\AxiomC{$\mathcal{T}( f(T(\alpha \land \beta),l))$}\UnaryInfC{$\mathcal{T}(L)$}\DisplayProof$ 
       $=\AxiomC{$\mathcal{T}( (L-T(\alpha \land \beta))||\alpha \land \beta||T \alpha||T \beta)$}\UnaryInfC{$\mathcal{T}(L)$}\DisplayProof$
       
       
       $=\AxiomC{$\Gamma ,\alpha, \beta, \alpha \land \beta \vdash \Delta$}\UnaryInfC{$\Gamma, \alpha \land \beta \vdash \Delta$}\DisplayProof$ 
       $=\RightLabel{\scriptsize{$\land$L}}\AxiomC{$\Gamma ,\alpha, \beta \vdash \Delta$}\UnaryInfC{$\Gamma, \alpha, \beta, \alpha \land \beta \vdash \Delta$}\UnaryInfC{$\Gamma, \alpha \land \beta \vdash \Delta$}\DisplayProof$
       \item 
       $\AxiomC{$\mathcal{T}( f(F(\alpha \land \beta),l))$}\UnaryInfC{$\mathcal{T}(L)$}\DisplayProof$ 
       $=\AxiomC{$\mathcal{T}( L - \sigma|| \sigma || F \alpha) $}\AxiomC{$\mathcal{T}( L - \sigma|| \sigma || F \beta) $}\BinaryInfC{$\mathcal{T}(L)$}\DisplayProof$
       
       
       $=\AxiomC{$\Gamma ,\alpha, \alpha \land \beta \vdash \Delta$}\AxiomC{$\Gamma , \beta, \alpha \land \beta \vdash \Delta$}\BinaryInfC{$\Gamma,\vdash  \alpha \land \beta  ,\Delta$}\DisplayProof$
       $=\RightLabel{\scriptsize{$\land$R}}\AxiomC{$\Gamma ,\alpha, \alpha \land \beta \vdash \Delta$}\AxiomC{$\Gamma , \beta, \alpha \land \beta \vdash \Delta$}\BinaryInfC{$\Gamma \vdash  \alpha \land \beta ,\alpha \land \beta, \Delta$}\UnaryInfC{$\Gamma,\vdash  \alpha \land \beta  ,\Delta$}\DisplayProof$
       \item 
   $\AxiomC{$\mathcal{T}( f(T(\alpha \lor \beta),l))$}\UnaryInfC{$\mathcal{T}(L)$}\DisplayProof$ 
   $=\AxiomC{$\mathcal{T}( L - \sigma|| \sigma || T \alpha)$}\AxiomC{$\mathcal{T}( L - \sigma|| \sigma || T \beta)$}\BinaryInfC{$\mathcal{T}(L)$}\DisplayProof$
   
   
   $=\AxiomC{$\Gamma ,\alpha , \alpha \lor \beta \vdash \Delta$}\AxiomC{$\Gamma ,\beta, \alpha \lor \beta \vdash \Delta$}\BinaryInfC{$\Gamma, \alpha \lor \beta \vdash \Delta$}\DisplayProof$
   $=\RightLabel{\scriptsize{$\lor$L}} \AxiomC{$\Gamma ,\alpha \vdash  \alpha \lor \beta,\Delta$}\AxiomC{$\Gamma , \beta, \alpha \lor \beta \vdash \Delta$}\BinaryInfC{$\Gamma \vdash  \alpha \lor \beta,\alpha \lor \beta ,\Delta$}\UnaryInfC{$\Gamma \vdash  \alpha \lor \beta ,\Delta$}\DisplayProof$
   \item 
   $\AxiomC{$\mathcal{T}( f(F(\alpha \lor \beta),l))$}\UnaryInfC{$\mathcal{T}(L)$}\DisplayProof$ 
   $=\AxiomC{$\mathcal{T}( (L-F(\alpha \lor \beta))||F \alpha||F \beta)$}\UnaryInfC{$\mathcal{T}(L)$}\DisplayProof$
   
   
   $=\AxiomC{$\Gamma \vdash \alpha \lor \beta,\alpha,\beta,\Delta$}\UnaryInfC{$\Gamma \vdash \alpha \lor \beta, \Delta$}\DisplayProof$
   $=\RightLabel{\scriptsize{$\land$R}}\AxiomC{$\Gamma \vdash \alpha \lor \beta,\alpha,\beta,\Delta$}\UnaryInfC{$\Gamma \vdash \alpha \lor \beta,\alpha \lor \beta,\alpha,\beta,\Delta$}\UnaryInfC{$\Gamma \vdash \alpha \lor \beta, \Delta$}\DisplayProof$
   \item 
   
   $\AxiomC{$\mathcal{T}( f(T(\alpha \rightarrow \beta),l))$}\UnaryInfC{$\mathcal{T}(L)$}\DisplayProof$  
   $=\AxiomC{$\mathcal{T}( L - \sigma|| \sigma || F \alpha )$}\AxiomC{$\mathcal{T}( L - \sigma|| \sigma || T \beta ) $}\BinaryInfC{$\mathcal{T}(L)$}\DisplayProof$  
   
   
   $=\AxiomC{$\Gamma, \alpha, \alpha \rightarrow \beta \vdash \Delta$}\AxiomC{$\Gamma, \alpha \rightarrow \beta \vdash  \beta, \Delta$}\BinaryInfC{$\Gamma, \alpha \rightarrow \beta \vdash \Delta$}\DisplayProof$  
   $=\RightLabel{\scriptsize{$\rightarrow$L}}\AxiomC{$\Gamma, \alpha, \alpha \rightarrow \beta \vdash \Delta$}\AxiomC{$\Gamma, \alpha \rightarrow \beta \vdash  \beta, \Delta$}\BinaryInfC{$\Gamma, \alpha \rightarrow \beta,\alpha \rightarrow \beta  \vdash \Delta$}\UnaryInfC{$\Gamma, \alpha \rightarrow \beta \vdash \Delta$}\DisplayProof$  
   
   \item 
   $\AxiomC{$\mathcal{T}( f(F(\alpha \rightarrow \beta),l))$}\UnaryInfC{$\mathcal{T}(L)$}\DisplayProof$  
   $=\AxiomC{$\mathcal{T}((L - F(\alpha \rightarrow \beta)) || F(\alpha \rightarrow \beta) || T \alpha || F \beta)$}\UnaryInfC{$\mathcal{T}(L)$}\DisplayProof$  
   
   
   $=\AxiomC{$\Gamma , \alpha \vdash \beta, \alpha \rightarrow \beta, \Delta$}\UnaryInfC{$\Gamma \vdash \alpha \rightarrow \beta, \Delta$}\DisplayProof$
   $=\RightLabel{\scriptsize{$\rightarrow$R}}\AxiomC{$\Gamma , \alpha \vdash \beta, \alpha \rightarrow \beta, \Delta$}\UnaryInfC{$\Gamma \vdash \alpha \rightarrow \beta, \alpha \rightarrow \beta, \Delta$}\UnaryInfC{$\Gamma \vdash \alpha \rightarrow \beta, \Delta$}\DisplayProof$
   
   \item 
   $\AxiomC{$\mathcal{T}( f(T(\forall x \, \phi(x)),l))$}\UnaryInfC{$\mathcal{T}(L)$}\DisplayProof$  
   $=\AxiomC{$\mathcal{T}( L - \sigma|| \sigma || T \phi(c_i) )$}\UnaryInfC{$\mathcal{T}(L)$}\DisplayProof$  
   
   
   $=\AxiomC{$\Gamma, \forall x \, \phi(x), \phi(c_i) \vdash \Delta$}\UnaryInfC{$\Gamma, \forall x \, \phi(x) \vdash \Delta$}\DisplayProof$  
   $=\RightLabel{\scriptsize{$\forall$L}}\AxiomC{$\Gamma, \forall x \, \phi(x), \phi(c_i) \vdash \Delta$}\UnaryInfC{$\Gamma, \forall x \, \phi(x), \forall x \, \phi(x) \vdash \Delta$}\UnaryInfC{$\Gamma, \forall x \, \phi(x) \vdash \Delta$}\DisplayProof$
   
   and $c_i$ does not occour in $\Gamma$ or $\Delta$, as it is a new one.
   \item 
   $\AxiomC{$\mathcal{T}( f(F(\forall x \, \phi(x)),l))$}\UnaryInfC{$\mathcal{T}(L)$}\DisplayProof$  
   $=\AxiomC{$\mathcal{T}( L - \sigma|| \sigma || F \phi(c_i) )$}\UnaryInfC{$\mathcal{T}(L)$}\DisplayProof$  
   
   
   $=\AxiomC{$\Gamma \vdash \phi(c_i), \forall x \, \phi(x), \Delta$}\UnaryInfC{$\Gamma \vdash \forall x \, \phi(x), \Delta$}\DisplayProof$  
   $=\RightLabel{\scriptsize{$\forall$R}}\AxiomC{$\Gamma \vdash \phi(c_i), \forall x \, \phi(x), \Delta$}\UnaryInfC{$\Gamma \vdash \forall x \, \phi(x), \forall x \, \phi(x), \Delta$}\UnaryInfC{$\Gamma \vdash \forall x \, \phi(x), \Delta$}\DisplayProof$
   
   \item 
   $\AxiomC{$\mathcal{T}( f(T(\exists x \, \phi(x)),l))$}\UnaryInfC{$\mathcal{T}(L)$}\DisplayProof$  
   $=\AxiomC{$\mathcal{T}( L - \sigma|| \sigma || T \phi(c_i) )$}\UnaryInfC{$\mathcal{T}(L)$}\DisplayProof$  
   
   
   $=\AxiomC{$\Gamma, \exists x \, \phi(x), \phi(c_i) \vdash \Delta$}\UnaryInfC{$\Gamma, \exists x \, \phi(x) \vdash \Delta$}\DisplayProof$  
   $=\RightLabel{\scriptsize{$\exists$L}}\AxiomC{$\Gamma, \exists x \, \phi(x), \phi(c_i) \vdash \Delta$}\UnaryInfC{$\Gamma, \exists x \, \phi(x), \exists x \, \phi(x) \vdash \Delta$}\UnaryInfC{$\Gamma, \exists x \, \phi(x) \vdash \Delta$}\DisplayProof$
   
   and $c_i$ does not occour in $\Gamma$ or $\Delta$, as it is a new one.
   \item 
   $\AxiomC{$\mathcal{T}( f(F(\exists x \, \phi(x)),l))$}\UnaryInfC{$\mathcal{T}(L)$}\DisplayProof$  
   $=\AxiomC{$\mathcal{T}( L - \sigma|| \sigma || F \phi(c_i) )$}\UnaryInfC{$\mathcal{T}(L)$}\DisplayProof$  
   
   
   $=\AxiomC{$\Gamma \vdash \phi(c_i), \exists x \, \phi(x), \Delta$}\UnaryInfC{$\Gamma \vdash \exists x \, \phi(x), \Delta$}\DisplayProof$  
   $=\RightLabel{\scriptsize{$\exists$R}}\AxiomC{$\Gamma \vdash \phi(c_i), \exists x \, \phi(x), \Delta$}\UnaryInfC{$\Gamma \vdash \exists x \, \phi(x), \exists x \, \phi(x), \Delta$}\UnaryInfC{$\Gamma \vdash \exists x \, \phi(x), \Delta$}\DisplayProof$
   
   \end  {itemize}
   }
       
   
   \end{proof}
}

\def\TranslationClassical{
\begin{definition} \textbf{Tree Translating Function}
    Given a tableaux proof tree development and its root r $\tau$, we define $\mathcal{T}_p(\tau)$:
    \begin{itemize}
        \item if $\tau$ = r and there are $T \sigma $ and $F \sigma $ on r:
        
        $\mathcal{T}_p(\tau) = \RightLabel{\scriptsize{Ax $\sigma$}}\AxiomC{}\UnaryInfC{$\mathcal{T}(r)$}\DisplayProof$
        \item if $\tau$ = r and there is no $\sigma$ such that $T \sigma $ and $F \sigma $ are on r: 
        
        $\mathcal{T}_p(\tau) = \mathcal{T}(\tau) $

        \item if $r$ has a single child $r_0$ with a corresponding subtree $\tau_0$: \\by definition, it exits a sentence $\sigma$ such that  
        $f(\sigma,r)=f_{Rule}(\sigma,r)=r_0$, and so 
        
        
        $\mathcal{T}_p(\tau) = \RightLabel{\scriptsize{Rule on  $\sigma$}}\AxiomC{$\mathcal{T}_p(\tau_0)$}\UnaryInfC{$\mathcal{T}(r)$}\DisplayProof$. 
        \item if $r$ has two children $r_1$ and $r_2$ with corresponding subtrees $\tau_1$ and $\tau_2$:\\ by definition there exists a sentence $\sigma$  such that
            $f(\sigma, r) = f_{Rule}(\sigma, r) = \{r_1, r_2\}$, and so:


            $\mathcal{T}_p(\tau) = \RightLabel{\scriptsize{Rule on $\sigma$}}\AxiomC{$\mathcal{T}_p(\tau_1)$}\AxiomC{$\mathcal{T}_p(\tau_2)$}\BinaryInfC{$\mathcal{T}(r)$}\DisplayProof$ 
    \end{itemize}
\end{definition}
}

\def\TranslationClassicalProof{
    \begin{theorem}
        given a classical tableaux development $\tau$, $\mathcal{T}_p(\tau)$ is a valid sequent proof 
    \end{theorem}
    \begin{proof}
        The proof goes by induction on the size of $\tau$:
        \begin{itemize}
            \item if $\tau$ = r then there are $T \sigma $ and $F \sigma $ on r, $\mathcal{T}_p(\tau) = \RightLabel{\scriptsize{Ax $\sigma$}}\AxiomC{}\UnaryInfC{$\mathcal{T}(r)$}\DisplayProof$ is a valid sequent proof.
            \item if $r$ has a single child $r_0$ with a corresponding subtree $\tau_0$: 
             $\mathcal{T}_p(\tau) = \RightLabel{\scriptsize{Rule on  $\sigma$}}\AxiomC{$\mathcal{T}_p(\tau_0)$}\UnaryInfC{$\mathcal{T}(r)$}\DisplayProof$
              is a valid sequent proof since $\mathcal{T}_p(\tau_0)$ is valid by induction hypothesis and the rule is valid by the last theorem.
            \item if $r$ has two children $r_1$ and $r_2$ with corresponding subtrees $\tau_1$ and $\tau_2$:\\ 
                $\mathcal{T}_p(\tau) = \RightLabel{\scriptsize{Rule on $\sigma$}}\AxiomC{$\mathcal{T}_p(\tau_1)$}\AxiomC{$\mathcal{T}_p(\tau_2)$}\BinaryInfC{$\mathcal{T}(r)$}\DisplayProof$ 
                is a valid sequent proof since $\mathcal{T}_p(\tau_1)$ and $\mathcal{T}_p(\tau_2)$ are valid by induction hypothesis and the rule is valid by the last theorem.

        \end{itemize}
    
    \end{proof}
}


\def\ThinningFunctionDefinition{
\begin{definition} \textbf{Tree Thinning Function}
    Given a tableaux proof tree development and its root r , we define $\mathcal{F}(\tau)$:
    \begin{itemize}
        \item If $\tau$ = r and there are $T_f \sigma $ and $F_f \sigma $ on r:
        
        $\mathcal{F}(\tau) = [T_f \sigma   , F_f \sigma ]$
 

        \item If $r$ has a single child $f(\sigma,r)$ :
        
        with a corresponding subtree $\tau_0$ and  $r'_0$ is the root of $\mathcal{F}(\tau_0)$:


         $(f(\sigma,r) - r) \cap r'_0 $ are the elements generated by the inference f that are necessary to prove the non-existance of a frame that respects $f(\sigma,r)$. In this case:
         \begin{itemize}
        \item
         If $f(\sigma,r) - r) \cap r'_0 \neq \emptyset $
         
         $\mathcal{F}(\tau) $ = a tree with root $(r'_0 - f(\sigma,\sigma))||\sigma$ connected to the child $r'_0 ||(f(\sigma,\sigma)-\sigma)$,  that has as a child the subtree $\mathcal{F}(\tau_0)$
         \item
         If $f(\sigma,r) - r) \cap r'_0 = \emptyset $

         
         $\mathcal{F}(\tau) $ = $\mathcal{F}(\tau_0)$
         \end{itemize}


         \item if $r$ has the children $f(\sigma,r)[1]$ and $f(\sigma,r)[2]$ :
         
         with the corresponding subtrees $\tau_1$ and $\tau_2$,  $r'_1$ is the root of $\mathcal{F}(\tau_1)$ and $r'_2$ is the root of $\mathcal{F}(\tau_2)$, 
         then $(f(\sigma,r) - r) \cap (r'_1 \cup r'_2) )$ also defines if f was "usefull":
         
         \begin{itemize}
            \item
             If $(f(\sigma,r) - r) \cap (r'_1 \cup r'_2 ) \neq \emptyset $:
             
             $\mathcal{F}(\tau) $ = a tree with root $r'_0||\sigma$ connected to the children $r'_1 ||(f(\sigma,\sigma)[1]-r)$  $r'_2 ||(f(\sigma,\sigma)[2]-r)$, each having 
             $\mathcal{F}(\tau_1)$ and $\mathcal{F}(\tau_2)$, respectivelly, bellow them. 
             \item
             If $(f(\sigma,r) - r) \cap (\mathcal{F}(r'_1) \cup \mathcal{F}(r'_2) ) = \emptyset $:
    
             
             $\mathcal{F}(\tau) $ = $\mathcal{F}(\tau_1) = \mathcal{F}(\tau_1)$
        \end{itemize}
    

    \end{itemize}
\end{definition}
}



\def\nFProof{
    \begin{theorem}
       (proof/search for a counter ex) If $\tau$ does not branch, the root $r'$ of $\mathcal{F}(\tau)$  has not more than 2 signed sentences $F$  
    \end{theorem}
    \begin{proof}

        
        
        The proof goes by induction on the size of $\tau$:
        \begin{itemize}
            \item If $\tau$ = r, $\mathcal{F}(\tau) = [T_f \sigma   , F_f \sigma ]$.  The theorem holds
     
    
            \item If $r$ has a single child $f(\sigma,r)$ :
            
            with a corresponding subtree $\tau_0$ and  $r'_0$ is the root of $\mathcal{F}(\tau_0)$:
    
    
            \begin{itemize}
            \item
             If $f(\sigma,r) - r) \cap r'_0 \neq \emptyset $:
             
             the root of $\mathcal{F}(\tau) $ is $(r'_0 - f(\sigma,\sigma))||\sigma$ 
             $r'_0$ has at most one $F$, so the only way for $(r'_0 - f(\sigma,\sigma))||\sigma$ to have more than 1 signed sentence of type is :
             
             if the other signed sentence in $r'$ is of type F (we denote $F_{q} B$) 
             
             
             and  $f = f_{F \neg } (or f_{F \to })$ that is, 
             
             
             $\sigma$ = $F_p\neg \alpha$. This would mean that $f(\sigma,\sigma) = T_{p'} \alpha$ for a new $p'$.  As $p'$ is 'new' it can not be $\leq q$ 
             so 
            
             $$

             \item
             If $f(\sigma,r) - r) \cap r'_0 = \emptyset $
    
             
             $\mathcal{F}(\tau) $ = $\mathcal{F}(\tau_0)$
             \end{itemize}
        
    
        \end{itemize}
    
    \end{proof}
}











