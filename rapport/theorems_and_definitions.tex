
\def\structureDefinition{
\begin{definition}
    \textbf{A Structure of a Language} consists of a domain and:
    \begin{itemize}
        \item An assignment from the constant symbols of the language to the domain.
        \item An assignment from the predicate symbols of the language to predicates in the domain.
    \end{itemize}
\end{definition}
}

\def\kripkeFrameDefinition{
\begin{definition}
    \textbf{A Kripke Frame} of a Language \(\mathcal{L}\), \(\mathcal{C} = (R, \{C(p)\}_{p \in R})\) consists of a partially ordered set \(R\), and an \(\mathcal{L}\)-structure \(C(p)\) for all \(p\)'s in \(R\). Furthermore, in a Kripke Frame, if \(p \leq q\), then \(C(q)\) extends \(C(p)\):
    \begin{itemize}
        \item All sentences that are true in \(C(p)\) are true in \(C(q)\).
        \item The domain of \(C(p)\) is included in the domain of \(C(q)\).
        \item The assignments in \(C(p)\) are the same as in \(C(q)\) for the domain in common.
    \end{itemize}
\end{definition}
}

\def\forcingDefinition{
\begin{definition}
    \textbf{Forcing}. When a sentence \(\phi\) of a language \(\mathcal{L}\) is \textbf{forced} by a structure \(C(p)\) of a frame \(\mathcal{C}\), we denote:
    \(p \Vdash_{\mathcal{C}} \phi\) \\
    Forcing is defined by induction: \cite{book1}
    \begin{itemize}
        \item \(p \Vdash_{\mathcal{C}} \phi \Leftrightarrow \phi \text{ is true in } C(p)\) (if \(\phi\) is an atomic sentence).
        \item \(p \Vdash_{\mathcal{C}} (\phi \to \psi) \Leftrightarrow\) for all \(q \geq p\), if \(q \Vdash_{\mathcal{C}} \phi\), then \(q \Vdash_{\mathcal{C}} \psi\).
        \item \(p \Vdash_{\mathcal{C}} \neg \phi \Leftrightarrow\) for all \(q \geq p\), \(q\) does not force \(\phi\).
        \item \(p \Vdash_{\mathcal{C}} (\forall x) \phi(x) \Leftrightarrow\) for all \(q \geq p\) and \(d\) in \(\mathcal{L}_{C(q)}\), \(q \Vdash_{\mathcal{C}} \phi(d)\).
        \item \(p \Vdash_{\mathcal{C}} (\exists x) \phi(x) \Leftrightarrow\) there exists a \(d\) in \(\mathcal{L}_{C(q)}\), such that \(p \Vdash_{\mathcal{C}} \phi(d)\).
        \item \(p \Vdash_{\mathcal{C}} (\phi \land \psi) \Leftrightarrow p \Vdash_{\mathcal{C}} \phi \text{ and } p \Vdash_{\mathcal{C}} \psi\).
        \item \(p \Vdash_{\mathcal{C}} (\phi \lor \psi) \Leftrightarrow p \Vdash_{\mathcal{C}} \phi \text{ or } p \Vdash_{\mathcal{C}} \psi\).
    \end{itemize}
\end{definition}
}

\def\intuitionisticValidityDefinition{
\begin{definition}
    \textbf{Intuitionistic Validity}. A sentence of a language \(\mathcal{L}\) is Intuitionistically valid if it is forced in all structures of all Kripke frames of \(\mathcal{L}\).
\end{definition}
}

\def\classicalValidityDefinition{
\begin{definition}
    \textbf{Classical Validity}. A sentence of a language \(\mathcal{L}\) is classically valid if it is forced by all single-structure Kripke frames of that sentence's language.
\end{definition}
}
